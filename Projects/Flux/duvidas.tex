%% LyX 1.2 created this file.  For more info, see http://www.lyx.org/.
%% Do not edit unless you really know what you are doing.
\documentclass[a4paper,brazil]{article}
\usepackage[T1]{fontenc}
\usepackage[brazil]{inputenc}

\makeatletter

%%%%%%%%%%%%%%%%%%%%%%%%%%%%%% LyX specific LaTeX commands.
\providecommand{\LyX}{L\kern-.1667em\lower.25em\hbox{Y}\kern-.125emX\@}

\usepackage{babel}
\makeatother
\begin{document}

\section{DeleteObject()}

\begin{enumerate}
\item Quando e quem ter� acesso ou precisar� deste m�todo ?\\
1. Quem -> Para mim s� pode ser uma tarefa que por algum motivo decide
apagar o(s) dado(s). Isso levanta uma nova quest�o, como isso � feito,
direto ou atrav�s do RequestDelete(), ou seja, DeleteObject() ser�
p�blico \\
2. Quando. N�o fa�o ideia.
\item Como ser� feito este acesso.\\
A estrutura (na atual implementa��o) permite que cheguemos nos MetaData
por v�rios caminhos, qual deles deve ser usado para invocar funcionalidades
de tal impacto como a dele��o de um objeto ?\\
1. Poderia chamar via DM (que eu acho errado !)\\
2. Poderia chamar via lista de depend�ncias quando numa tarefa (aposto
nessa)
\end{enumerate}
Um exemplo para tentar identificar estas necessidades !

Uma tarefa usa um dado como uma vari�vel tempor�ria que ser� distribuida
no ambiente. Depois de sua utiliza��o, a tarefa pode submeter um pedido
de dele��o deste objeto. Como isso seria feito ? Digamos que o Id
do dado tempor�rio � guardado pela tarefa. Quando esta n�o mais utilizar
o dado, e estando certo que nenhuma outra tarefa utilizar� o dado,
este poder� ser exclu�do do DM.

Uma vez que n�o existe requisi��o de acesso de nenhuma tarefa sobre
o dado, a dele��o � imediata, caso contr�rio, inicia-se a solicita��o
de dele��o do dado.


\section{AmIConsistent()}

N�o entendo qual o erro no m�todo. Se eu (DataDepend) necessito de
um dado em uma certa vers�o e este n�o mais voltar� a esta vers�o,
acho que sou inconsistente n�o ?
\end{document}
